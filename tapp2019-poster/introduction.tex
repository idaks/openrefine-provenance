\section{Introduction}

Preparation of data sets for analysis is a critical component of research in many disciplines. Recording the steps taken to clean data sets consequently is essential to making such research transparent and results reproducible. OpenRefine is a tool for interactively cleaning data sets via a spreadsheet-like interface \cite{verborgh_using_2013}. OpenRefine records the sequence of operations carried out and changes made to a data set. It uses these records to provide an undo/redo capability that allows a user to revisit the past state of the data set at any point following the initial data import step. OpenRefine further enables the user to export sequences of these recorded operations as \emph{recipes} that can be saved and applied later to different data sets.  However, because such exported recipes do not include edits made manually to individual cells, a recipe cannot represent the end-to-end data preparation workflow. Here we investigate how the complete operation history recorded by OpenRefine can be used both to facilitate reproduction of complete, real-world data cleaning workflows and to support queries and visualizations of the provenance of cleaned data sets.

