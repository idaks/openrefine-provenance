\section{Introduction}

Preparation of data sets for analysis is a critical component of research in many disciplines. Recording the steps taken to clean data sets is equally crucial if such research is to be transparent and results reproducible. OpenRefine is a tool for interactively cleaning data sets via a spreadsheet-like interface and for recording the sequence of operations carried out by the user \cite{verborgh_using_2013}. OpenRefine uses its operation history to provide an undo/redo capability that enables a user to revisit the state of the data set at any point in the data cleaning process. OpenRefine additionally allows the user to export sequences of recorded operations as \emph{recipes} that can be applied later to different data sets. Although OpenRefine records details about every change made to a data set, exported recipes do not include edits made manually to individual cells. Consequently, neither a single recipe, nor a set of recipes, can in general represent an entire, end-to-end data preparation workflow. 

Here we report early results from an investigation into how the operation history recorded by OpenRefine can be used to (1) facilitate reproduction of complete, real-world data cleaning workflows; and (2) support queries and visualizations of the provenance of cleaned data sets for easy review.
