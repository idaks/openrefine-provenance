% start of document preamble

%%%% Proceedings format for most of ACM conferences (with the exceptions listed below) and all ICPS volumes.
\documentclass[sigconf,screen,nonacm]{acmart}

% defining the \BibTeX command - from Oren Patashnik's original BibTeX documentation.
\def\BibTeX{{\rm B\kern-.05em{\sc i\kern-.025em b}\kern-.08emT\kern-.1667em\lower.7ex\hbox{E}\kern-.125emX}}
 
\usepackage[english]{babel}
\usepackage{enumitem}
\setlist[itemize]{leftmargin=3 mm}

% start of the body of the document body.
\begin{document}

% title
% \title[Reproducible Data Cleaning]{Poster Abstract: Reproducible Data Cleaning Workflows Provenance Query and Visualization for OpenRefine}
\title[Reproducible Data Cleaning]{POSTER: Modeling
  Provenance and Understanding Reproducibility for OpenRefine Data
  Cleaning Workflows}

\author{Timothy McPhillips \qquad Lan Li \qquad Nikolaus Parulian \qquad Bertram Lud\"ascher}
% BL: The braces don't seem to work ... 
%\email{\{tmcphill,lanl2,nnp2,ludaesch\}@illinois.edu}
 
% BL: so let's do some manual hacking: 
\affiliation{%
  \institution{School of Information Sciences,  University of Illinois at Urbana-Champaign \\ 
    \texttt{\{tmcphill,lanl2,nnp2,ludaesch\}@illinois.edu} }
}

% This command processes the author and affiliation and title information and builds the first part of the formatted document.
\maketitle

\section{Introduction}

Preparation of data sets for analysis is a critical component of research in many disciplines. Recording the steps taken to clean data sets is equally crucial if such research is to be transparent and results reproducible. OpenRefine is a tool for interactively cleaning data sets via a spreadsheet-like interface and for recording the sequence of operations carried out by the user \cite{verborgh_using_2013}. OpenRefine uses its operation history to provide an undo/redo capability that enables a user to revisit the state of the data set at any point in the data cleaning process. OpenRefine additionally allows the user to export sequences of recorded operations as \emph{recipes} that can be applied later to different data sets. Although OpenRefine records details about every change made to a data set, exported recipes do not include edits made manually to individual cells. Consequently, a recipe in cannot generally represent an entire, end-to-end data preparation workflow. 

Here we report early results from an investigation into how the operation history recorded by OpenRefine can be used to (1) facilitate reproduction of complete, real-world data cleaning workflows; and (2) support queries and visualizations of the provenance of cleaned data sets for easy review.

\section{Aims}

The results described here represent initial steps in our efforts to:

\begin{itemize}[label=\raisebox{0.25ex}{\tiny$\bullet$}]

\item Understand and describe the native data and history model of OpenRefine using the concepts and terminologies of the reproducible-research and provenance communities.

\item Discover what provenance queries can be supported by the OpenRefine data model and operation history.  Demonstrate queries that reveal key aspects of the provenance of cleaned data sets.

\item Extend the YesWorkflow process, data, and provenance models as needed to represent the operations, transformations, data structures, data flows, and data dependencies that characterize data cleaning workflows.  

\item Employ YesWorkflow to represent end-to-end workflows carried out using OpenRefine so that they can be visualized readily and queried prospectively.

\item Identify provenance queries important for achieving research transparency that apparently \emph{cannot} be satisfied using just the information recorded by OpenRefine. Develop means to augment the operation history with additional information needed to support these critical queries.

\item Employ computational environments that can be reproduced reliably across multiple computer systems maintained by different researchers. Enable other members of the community independently to repeat our experiments and demonstrations, and to review and reproduce our results on their own computers.

\end{itemize}

\section{Tools}

We use the following tools. We run OpenRefine version 3.1 \cite{OpenRefine} in Java 8 environments on multiple platforms. We access the OpenRefine HTTP API by using and extending the OpenRefine Python Client Library \cite{makepeace18ORclient}. We use the YesWorkflow toolkit \cite{yw-website} for modeling data cleaning workflows and representing OpenRefine operation histories in queryable form. We employ XSB Prolog \cite{xsb_software} for expressing and performing Datalog-style graph and provenance queries, and Graphviz for rendering visualizations of query results. We use GitHub to share research artifacts between coauthors and with the community. We depend on Ansible, Vagrant, and Docker for making research environments reproducible across coauthors' computers and for enabling other researchers to repeat our experiments on their own systems. Finally, we share preconfigured computing environments for reproducing our results using resources provided by the Whole Tale and MyBinder projects.



%\section{Example Provenance Questions}

%Let's do some \emph{italics} and \texttt{teletype} and \textsf{sans
%  serif}. Does it work?
%Here's a bit of code, very simple via \verb|some verbatim| in line, or
%\begin{verbatim}
%.. in some separate verbatim paragraphs ..
%Does it work?
%   Even indentation should work .. 
%\end{verbatim}

\section{Results}

We demonstrate that under certain conditions complete data cleaning workflows carried out within OR can be repeated fully automatically in a different instance of OR.  We shown that this requires using information that is recorded by OR for its undo/redo feature, but that is not exportable from OR via recipes or its native HTTP API.
We show that key queries of the provenance of cleaned data sets, and of particular columns, rows, and cells in the final data set, can be satisfied using the information captured by OR for its undo/redo feature.
Will illustrate the usefulness of YesWorkflow-style workflow diagrams for making data cleaning workflows transparent and easy to review, and for rendering portions of the overall workflow to represent the result of provenance queries.

\section{Future Work}


% \begin{acks}
% Work supported in part by NSF award 1541450 (Whole Tale).
% \end{acks}

%
% The next two lines define the bibliography style to be used, and the bibliography file.
\bibliographystyle{alpha-initials-big}
\bibliography{main}

\end{document}
